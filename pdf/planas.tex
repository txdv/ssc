\documentclass{VUMIFPSbakalaurinis}
\usepackage{algorithmicx}
\usepackage{algorithm}
\usepackage{algpseudocode}
\usepackage{amsfonts}
\usepackage{amsmath}
\usepackage{bm}
\usepackage{caption}
\usepackage{color}
\usepackage{float}
\usepackage{graphicx}
\usepackage{listings}
\usepackage{subfig}
\usepackage{wrapfig}

% Titulinio aprašas
\university{Vilniaus universitetas}
\faculty{Matematikos ir informatikos fakultetas}
\institute{Informatikos institutas}  % Užkomentavus šią eilutę - institutas neįtraukiamas į titulinį
\department{Programų sistemų bakalauro studijų programa}
\papertype{Bakalauro baigiamojo darbo planas}
\title{Kompiliatoriaus transliuojamo vieneto smulkinimas}
\titleineng{Translation unit granularization}
\author{Andrius Bentkus}
% \secondauthor{Vardonis Pavardonis}   % Pridėti antrą autorių
\supervisor{asist. dr. Vytauts Valaitis}
\reviewer{}
\date{Vilnius – \the\year}

% Nustatymai
% \setmainfont{Palemonas}   % Pakeisti teksto šriftą į Palemonas (turi būti įdiegtas sistemoje)
\bibliography{bibliografija}

\begin{document}
\maketitle

%% Padėkų skyrius
% \sectionnonumnocontent{}
% \vspace{7cm}
% \begin{center}
%     Padėkos asmenims ir/ar organizacijoms
% \end{center}

\begin{samepage}
%Planas
\section{Plan}
%Tyrimo objektas ir aktualumas
\subsection{Research object and actuality}
The majority of the research done on compilers is focused on making compilers output optimized programs in terms of code execution speed and memory usage \cite{lopes2018future} while neglecting or wilfully sacrificing \cite{fast2019compilers} actual runtime performance of the compiler itself.
As computers get faster more resources are available, but programmers tend to utilize these newfound resources to make the implementation of programs simpler rather than making the programs faster.

Another vector of academic improvement within the compiler sphere is to add new features or utilize new paradigms like dependent types or formal specifications in order to allow the compiler to do more sophisticated type and program checking than classic type checking allows, significantly increasing runtime compilation.

This is a new paragraph.

\subsection{Keliami uždaviniai ir laukiami rezultatai}
The goal of this bachelor thesis is to create a minimal compiler for a small subset of the Scala programming language.
Scala is known to have a multitude of advanced and complicated features{ScalaSpec}, however the targeted compiler implementation will support only the most basic once.

% Tyrimo Metodas
\subsection{Investigation method}
% Darbo atlikimo procesas
\subsection{Work process}
% Used literature
\subsection{Used literature}
\end{samepage}

\printbibliography[heading=bibintoc]  % Šaltinių sąraše nurodoma panaudota
% literatūra, kitokie šaltiniai. Abėcėlės tvarka išdėstomi darbe panaudotų
% (cituotų, perfrazuotų ar bent paminėtų) mokslo leidinių, kitokių publikacijų
% bibliografiniai aprašai. Šaltinių sąrašas spausdinamas iš naujo puslapio.
% Aprašai pateikiami netransliteruoti. Šaltinių sąraše negali būti tokių
% šaltinių, kurie nebuvo paminėti tekste. Šaltinių sąraše rekomenduojame
% necituoti savo kursinio darbo, nes tai nėra oficialus literatūros šaltinis.
% Jei tokių nuorodų reikia, pateikti jas tekste.

% \sectionnonum{Sąvokų apibrėžimai}

\appendix  % Priedai
% Prieduose gali būti pateikiama pagalbinė, ypač darbo autoriaus savarankiškai
% parengta, medžiaga. Savarankiški priedai gali būti pateikiami ir
% kompaktiniame diske. Priedai taip pat numeruojami ir vadinami. Darbo tekstas
% su priedais susiejamas nuorodomis.

\end{document}
